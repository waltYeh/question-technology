\documentclass{article} 
\usepackage{xeCJK}
\usepackage{cite}
\setCJKmainfont{Droid Sans Fallback}
\author{叶欣} 

\title{《技术的追问》词语分析} 
\begin{document}   
\maketitle   



\begin{abstract}

\end{abstract}



\tableofcontents

\section{引言} 

	\subsection{选题的意义} 
		\paragraph{}
过去的几百年时间是人类技术水平突飞猛进的阶段,在科技使得人们生活水平大幅度提高的同时,技术产品的生产和使用导致了环境破坏,技术因素的差距导致了全球经济发展的不平衡,技术的危险应用还在两次世界大战中夺去了无数生命。现代技术对人类历史进程的巨大冲击,引发了很多哲人的思考。纵观西方哲学,马克思、尼采、海德格尔先后都在处理现代性问题,都针对技术工业给人类带来的变动进行了深刻揭示。
		\paragraph{}
海德格尔是20世纪最伟大的思想家之一,他对技术问题的思考是他后期思想的重要组成部分。海德格尔对现代技术的本质做了存在历史和形而上学批判意义上的讨论,被认为是最晦涩的。针对技术本质的问题,海德格尔提出了座架(Gestell)这一概念,并在海德格尔哲学的语境下进行了深入探索。由于海德格尔语言晦涩,其写作中“座架”“本质”等概念较为抽象,有时词义被海德格尔拓展,有时词语被海德格尔溯源,并与其他词语相关联,往往令人难以理解。因此本研究认为,从分析德文词语词源、基本词义、词族关联等方面入手,是理解海德格尔技术之思的重要途径。
		\paragraph{}
海德格尔哲学,尤其是他关于技术的论述引发了中国学界的重视。在译介方面,以孙周兴为代表的学者翻译了海德格尔大量演讲与论文,对海德格尔思想表述中Ereignis、Dasein、Gestell等重点概念的含义和译法进行了深入探讨。在对海德格尔技术思想的解读方面,国内已有大量论著,如范玉刚的《睿思与歧误:一种对海德格尔技术之思的审美解读》,包国光的《海德格尔生存论视域下的技术》,研究论文更是多如牛毛。但是,从语言、词语的角度出发,对海德格尔技术思想重点概念所进行的研究却尚未看到,这不得不说是一个遗憾。
		\paragraph{}
此外,本研究尝试结合中、英文翻译用词进行研究,从词语角度辨析多种语言对海德格尔哲学的各自解读,能极大帮助我们理解德文词语的完整含意,并认识不同文化对海德格尔哲学的引介情况。
	
	\subsection{研究目的}
		\paragraph{}
        本课题拟通过对海德格尔《技术的追问》中核心词汇的深入解读,通过对Ge-stell(座架)和Wesen(本质)等词汇的词源、所属词族、中英文翻译的研究,结合海德格尔相关文章的用词背景、在本文中对词语关系的推演,以达到认识德语词语的完整含义、追踪海德格尔技术之思的路途、深入理解海德格尔所谓“技术本质”的目的,并为技术时代的生活提供借鉴。
    \subsection{问题的提出}
		\paragraph{}
     	在深入了解海德格尔技术思想之前,我们就已经在不经意间从大量介绍材料中对其略知一二。对于一个未曾深入认识海德格尔的人来说,提到他除了想到其名作《存在与时间》之外,就是想到他对技术的批判反思了。读过几篇介绍材料的话,会对海德格尔提到的“古代的风车”“莱茵河上的水坝”印象深刻,还会知道一个晦涩的词语“座架”,还知道这词可不是一个太好的东西。然而,海德格尔的思想实在艰深,尤其如果我们只阅读过中文翻译的文章或者中文写的介绍评析,那么想真正理解其技术思想更是难上加难,我们往往陷入于很多中文看起来互不相干的概念之中,不止所云。
     	\paragraph{}
     	如今,借助德文原文,我们能够深入一步。看到那些我们在基础德语中就学过的最基本的词语竟可以被用来描述如此高深的思想,不禁感叹这语言的魅力。于是更深刻地理解这些德文词语的含义,便成为了我们把握海德格尔技术思想的关键所在。追踪海德格尔《技术的追问》,我们也开始了自己的追问。
     	从词语角度出发,我们提出以下问题。
		\subparagraph{1} 
            通过哪些词语,我们能够把握海德格尔的技术思想?
		\subparagraph{2} 
           在海德格尔思想中,为何用stellen(摆置)解读现代技术?“技术的本质”中“本质”(Wesen)的含义是什么?
		\subparagraph{3}海德格尔为何提出Ge-stell(座架)一词?它在何种意义上是技术的本质?
		\subparagraph{4}中文、英文对德文词语的翻译能否为我们理解海德格尔提供线索?
		\subparagraph{5}技术的座架本质对身处技术时代的我们有何启示?
	\subsection{论文结构}
		\paragraph{}
        论文的第一部分为引言,介绍本课题的研究意义、目的,提出待研究的问题。
		\paragraph{}
        论文的第二部分,简要介绍海德格尔哲学思想、语言特点,对本课题研究对象《技术的追问》进行总体介绍,并对《技术的追问》中的重点词语及所属词族进行统计和分类,便于后续深入研究的开展。
        \paragraph{}
        论文的第三部分,拟分析stellen(摆置)这一核心词族的词源、用法和含义。理解海德格尔对stellen的论述,认识Gestell(座架)概念如何被提出。结合中英译文的处理方式进行理解。
        \paragraph{}
        论文的第四部分,拟分析“座架作为技术之本质(Wesen)”中,本质一词的词源、用法和含义。区分技术本质与技术性的东西。探索währen、gewähren、Wahrheit等词语是如何从德语角度和海德格尔的角度与Wesen联系起来的。揭示座架何以是技术之本质。
        \paragraph{}
        论文的第五部分为结语,对本文进行简要总结。还会提及海德格尔对其他词语的运用,作为对未来研究的展望。最后,反思技术的座架本质对技术时代的启示。
\section{海德格尔与《技术的追问》} 
	\subsection{《技术的追问》简介}
		\paragraph{}
		《技术的追问》来源于海德格尔的演讲。1949年12月1日,海德格尔在不莱梅进行了四个演讲,题目分别为“物(Das Ding)”“座架(Das Gestell)”“危险(Die Gefahr)”“转向(Die Kehre)”。1954到1955年期间上述第二个演讲被扩充,以“技术的追问(Die Frage nach der Technik)”为题,作为演讲系列“技术时代的艺术”的一部分。1962年“技术的追问”被收入《技术与转向》一书。
		\paragraph{}
		虽然海德格尔的技术思想并非集中而系统地在一部著作中阐述,但《技术的追问》依然是体现海德格尔技术思想的核心作品之一。《技术的追问》始终以探寻人与技术的自由关系为目标,以探究技术的本质(而非技术因素)为主线,走上了一条思之道路。
		\paragraph{}
		《技术的追问》开篇先强调了技术本质与技术因素的区分,然后否定了一般认为的工具性技术观,新提出了人作为四因之一的观点,得到技术本质的解蔽性质。之后文章集中描述了现代技术的促逼特性,举例说明技术如何摆置、订造自然,将之解蔽为持存物。探究上述行为的施行者而得到了“座架”概念作为技术的本质。之后从本质层面上的历史发生角度探究了自然科学与技术的关系,结合人与技术的关系得到座架的遣送特性。又由技术的促逼式的单一化解蔽方式得到座架的危险特性。作品此后从诗句出发探寻技术本质中的拯救,由本质一词联系到允诺,结合解蔽遣送,从而发现座架中蕴藏的拯救。最终,文章得到技术本质的两义性结论,提出了艺术创作对技术的救渡。
		\paragraph{}
		如此纵观全文,我们可以看到海德格尔对技术有褒有贬的两面,最终的结论也是强调技术的两义性。一方面是现代技术的促逼、摆置、遮蔽其他可能的危险,另一方面是本质上技术的解蔽、遣送、允诺、将要从中升起的救渡。但文章揭示这两面性质,走的却是一条连贯的追问道路。

	\subsection{海德格尔的技术思想}
		\paragraph{}
		技术问题是海德格尔思想的一个重要领域,海德格尔也是最早明确地从哲学角度探讨技术问题的哲学家之一。在海德格尔前期哲学代表作《存在与时间》中,就已经有一些对技术问题的探讨【(En)Framing Heidegger】,但对技术问题真正的深入,主要是在其后期思想之中。虽然其技术哲学并没有在一部如《存在与时间》一般的著作中系统阐述,但《世界图像的时代》《技术的追问》《转向》等演讲、文章中,都涉及到了技术问题。海德格尔的技术哲学思想相比我们常人对技术的认识,是别具一格的,这体现在诸多方面。
		\paragraph{}
		从研究方法上看,海德格尔是在他一贯的存在哲学的范畴内以现象学方法进行的探究。
		\paragraph{}
		现象学方法对于海德格尔来说,就是让事物以自己原初的方式显现,而非基于预先设定的理论认识事物。/cite{Understanding Heidegger on Technology}这种方法贯穿了海德格尔前期、后期的全部思想。以这种方法进行研究,与科学的研究方法形成了对立。科学是以一种固定的模式进行研究的,例如研究自然,科学的方法是:把自然当作一个先行可计算的力之关联体呈现出来,所以实验才被订造。(53段)那么如果用类似的方法探究技术,就不可能从技术之外把握技术的本质层面,只能看到技术因素(Technisches),如此导致的对技术的认识,海德格尔将之称为“正确的”而非“真实的”。只有通过现象学的方法,才可能找到真实的认识。而正是从现象学的方法,海德格尔提出“座架”概念。这种方法在45段已经明确被提及:Darum müssen wir auch jenes Herausfordern, ..., so nehmen, wie es sich zeigt.意思是说我们必须“如其所显示的那样”看待那种促逼。后来对座架的分析又反过来又揭示出了科学的研究方法存在的问题,因为现代科学也是为更原初的技术本质“座架”所发动。他说:Wo dieses (Ge-stell, Bestellen) herrscht, vertreibt es jede andere Möglichkeit der Entbergung. Vor allem verbirgt das Ge-stell jenes Entbergen, das im Sinne der ποίησις das Anwesende ins Erscheinen her-vor-kommen läßt.座架的这种订造占统治地位之处,它便驱除任何另一种解蔽的可能性。
		\paragraph{}
		现象学方法在《技术的追问》中的体现,就是海德格尔在分析过程中对语言的着重。海德格尔在文中对“技术”的词源、“本质”的词源和词语关联进行了重点探究,又借助荷尔德林的诗“但哪里有危险,哪里也有救”开始了在座架本质之中对拯救的寻找。
		\paragraph{}
与一些人从特定技术(如人工智能、克隆技术)出发的思考不同,海德格尔思考技术问题的角度是从本质层面进行的,并且他将技术的本质与技术性事物(技术因素)做出了区分。在《技术的追问》中,他说道:So ist denn auch das Wesen der Technik ganz und gar nichts Technisches.(3段)对此,他进一步解释道,如果我们仅仅追逐或表象技术因素,那么我们绝不能经验到我们与技术本质的关系。从技术的本质层面,海德格尔思得了“座架”概念,座架自然也不是技术性的事物或技术因素:Ge-stell heißt die Weise des Entbergens, die im Wesen der modernen Technik waltet und selber nichts Technisches ist.(51段)从本质角度思考技术使得海德格尔能够从现象学和存在的角度看技术,让技术自行显示其本质,座架的解蔽和遣送性质因此而成为可能。正如56段所说:在思想领域中有一种努力,就是更原初地去深思那种原初地所思的东西。Darum ist im Bereich des Denkens eine Bemühung, das anfänglisch Gedachte noch anfänglicher zu durchdenken.
		\paragraph{}
海德格尔的技术观中,人的地位也是独特的。一些人认为,人只有成为技术的主人,才有可能充分利用技术的优点,才有可能阻止技术向坏的方向发展。根据马克思主义的观点,技术特性是中立性与价值性的统一,技术是中立的工具和手段,技术为了特定目的而为人服务。海德格尔承认以上认识的正确性,却说:… Darum ist das bloß Richtige noch nicht das Wahre.技术的工具观不是“真实”的东西,它无法揭示技术的本质,因而不能给人带来与技术的自由关系。根据古老的四因说,海德格尔认为人只是技术呈现的四个因之一。从另一个方面看,技术的解蔽性质也不是人能够支配的(kein bloß menschliches Tun),人反而被技术促逼,参与技术订造。正是由于这种对人与技术关系的认识,使得海德格尔能够认识技术的本质,提出“座架”这一技术本质概念。
		\paragraph{}
海德格尔对技术与科学的关系认识可谓哥白尼式的转变。众所周知,科学特别是数学和现代物理学的出现,比工业革命技术大发展要早了上百年,那么技术就是被应用的自然科学吗?但是海德格尔从本质的角度再次用“正确”而非“真实”来评判科学先于技术的观点:Historisch gerechnet, bleibt dies richtig. Geschichtlich gedacht, trifft es nicht das Wahre.(54段)从“历史”上考虑,是要将gechichtlich一词从存在的遣送(schicken)以及发生(geschehen)意义上进行理解,而非从历史学(historisch,英译作chronological,年代学)来看问题。如此考虑,技术的本质“座架”是先于科学的,只是技术的本质一直遮蔽着自身,即使科学发展、技术突飞猛进,依然如此,所以才造成了科学先于技术的假象。之所以座架先于科学,是因为现代科学采取一种尺度,它把自然先行作为可计算的力之关联体来加以追逐,据此订造实验,摆置自然,带有“拷问”自然的意思。这种订造与摆置的方式,与后来的技术对待自然的方式如出一辙,都是由技术的本质“座架”所发动。

	\subsection{海德格尔的语言特点}
		\paragraph{}
		从海德格尔的多篇著述、各国学者对海德格尔作品的翻译、他人对海德格尔的评价来看,可以发现海德格尔的语言有其鲜明特点,主要包括:方向性强、严整性弱、重视词语的使用等。
		\paragraph{}
		海德格尔作品的方向性、顺序性强。其行文过程如同一次探索的经历。正如海德格尔在《技术的追问》开篇所说,“追问构筑一条道路。因此之故,我们大有必要首先关注一下道路,而不要牵挂于个别的句子和明目。”
\cite{slct_Hei_Szx}
基于这样一个基调,在海德格尔行文过程中,一个词语可以在文中多次出现,但后来出现时往往比之前含义更加丰富。比如在《技术的追问》中,“本质”一词开头就多次出现,但这时“技术本质”的含义只是以“某物所是的那个什么”这个程度上理解的,直到文章中后部,出现一句“直到现在,我们还是在流俗的含义上来理解‘本质’一词的”
\cite{slct_Hei_Szx}
,从这开始,本质一词的含义才被继续开发,并联系到“持续”“允诺”等词语。
在海德格尔的全集前言草稿中,他写道:“Wege – nicht Werke”(道路——而非著作)。在对此的解释中,他说他的全集是“在对多义的存在问题所作的变动不居的追问道路之野上的一种行进(Unter-wegs,在途中)”。
\cite{sprach_sein_Szx}
还可以注意到,道路一词使用了复数形式,这正印证了他的追问的“变动不居”,即海德格尔不认为他的思想会像古典哲学那样搭建一座形而上学的完整建筑,而是从多个方向探索,行进在思想道路上。这与海德格尔的教学习惯也有密切联系,他不希望学生们只是听从他的观点而已。他希望学生或读者在思之旅途上伴随他,引领学生修筑自己的思想道路。
		\paragraph{}
		正是由于海德格尔语言如同道路一般的特点,他的文章有时令人觉得不够严谨。伽达默尔说,“后期海德格尔自己为了逃避形而上学的语言而发展出他半诗性的特殊语言”
\cite{Text_Explain_DeFr}
,他甚至用“女巫式风格”来形容其后期著作
\cite[pg. 113]{Deonstr_DeFr}
。这一点在《技术的追问》中也有所体现。比如从währen(持续)到gewähren(允诺),海德格尔仅举歌德作品中一例用词,说歌德的“耳朵在此听出了两词之间的未曾道出的一致”,便下了结论“只有允诺者才持续”
。又比如他所说危险中的解救,是以“荷尔德林的诗句道出了真理”为前提的\cite{slct_Hei_Szx}。
海德格尔有时也以比喻代替论证,比如在《技术的追问》结尾使用的Konstellation(星座)一词。
		\paragraph{}
		海德格尔说过,“语言是存在之家”,对词语的重视是海德格尔思想的重要特点。在行文过程中,他有时探究词源、有时阐释词语日常用法、有时赋予词语新含义。如伽达默尔所说,海德格尔甚至喜欢把词语引回到它们已经失落了的不再具有的词义上去,并从这种所谓词源学的词义中得出结论
\cite{Deonstr_DeFr}。溯源的词语多为希腊文或德文。古希腊作为西方哲学的源头,其文字蕴藏着丰富的哲学遗产;而德文却往往能够被追溯到一个未被哲学概念干扰、未被拉丁文解读干扰的本源。在《技术的追问》中海德格尔就先后对两个核心词语Technik和Wesen分别进行了希腊文和德文的词源探寻。探究一个词族中词语的关联就不胜枚举了,比如stellen schicken hören wesen等词的词族。给词语赋予新含义的例子也很多,比如前期哲学中描述人的存在状态的烦(Sorge)、畏(Angst)、《技术的追问》中的座架(Ge-stell)等等。
\section{词语分析}
	\subsection{《技术的追问》中的重要词语}
		\paragraph{}
		上文已经提到,现象学方法在《技术的追问》中的体现,就是海德格尔在分析过程中对语言的着重。在《技术的追问》一文中,出现了繁多的被海德格尔多次提及,或被重点分析,或被赋予了特殊哲学含义的词语。可以说海德格尔是将他的思想全面灌注到了词语之中。因此对于我们读者来说,面对海德格尔的“女巫”式语言风格,把握词语的含义,以词语为路标来进行思考是非常有必要的。而首要的,是找到几个贯穿文章、体现海德格尔此文核心思想的核心词语。以这些词语为起点,由表及里,我们去分析它们在文中出现的位置和频率,分析出现它们的句子以及它们在句子中的成分,分析这些词语的词源以及海德格尔对它们的使用,最终能使得我们更准确而全面地了解海德格尔给这些词语赋予的哲学含义,更准确而全面地了解海德格尔的技术哲学思想。
		\paragraph{}
		我们大致可以将文中的重要词语分为几类,以便于我们从重要词语中进一步寻找核心的、值得分析的词语。首先是与此文的主题直接相关的,海德格尔在此文中追问技术,那么这个词自然是“技术(Technik)”。如果再进一步,看海德格尔是如何技术问题的,那么就能发现从始至终贯穿全文的“本质(Wesen)”一词,即海德格尔是从本质层面探讨技术的,根据海德格尔的阐述,与“本质”相关的词语还有“持续(währen)”“允诺(gewähren)”等等。海德格尔对于现代技术的著名举例描述中,大量出现了“摆置(stellen)”以及stellen词族的其他相关词语。从“摆置”,海德格尔提出了一个为众人所知的海氏技术哲学概念“座架(Ge-stell)”,此后的分析也是围绕这个概念进行的。除此之外,还有一些词在海德格尔的行文过程中非常重要,而且它们在文中也与它们所属的词族一起出现,使得这些词携带上了全词族的意义,比如“遣送(schicken)”以及与之相关的“命运(Geschick)”“历史(Geschichte)”,再比如说“归属(gehören in)”以及相关的“倾听者(Hörender)”“奴隶(Höriger)”。最后,还有一类词语,当我们在文中看到这些词的时候就确定无疑,这一定是海德格尔的作品,这些词贯穿了海德格尔的整体哲学思想,不仅是其技术思想中的概念,也是海德格尔思想的核心概念,它们包括“解蔽(Entbergen)”“发生(Ereignis,又译作本有、大道等等)”“在场(an-wesen)”“真理(Wahrheit)”“用(Brauch)”等等。
		\paragraph{}
		那么,哪些词语才是本文的核心词语呢?显而易见,是那些或贯穿全文,或被围绕着进行阐述的词语。因此可以看出,摆置、本质、座架这三个词是能够被归属于核心词语的。后面的词语分析将先后围绕这三个词语展开。
	\subsection{《技术的追问》词语统计与分类}
		\subsubsection{stellen词族统计分类}
			\paragraph{}
			stellen词族作为《技术的追问》的核心词语,在本文中出现138次(不计Ge-stell)。合并名词动词等词性的区分,仅从词语前缀与词义上进行区分,得到了以下数据。这里附上《朗氏德汉双解大词典》中对这些词的中文解释,这些词的某些含义海德格尔在《技术的追问》中并未使用,但它们的各种含义可以让我们对stellen这个词族的含义丰富性有一个大体了解。
			\subparagraph{stellen}
			28次。摆放、放置;拨置,调节;给…配置、提供;提出。
			\subparagraph{bestellen}
			53次。订货、订购;预订;预约、约请;耕种。
			\subparagraph{vorstellen}
			18次。向…介绍;推出、展示;设想。
			\subparagraph{feststellen}
			6次。查明、确定;看出;明确指出。
			\subparagraph{darstellen}
			5次。展示、再现;阐述、描述、说明;描绘、表现;意味着、表示;表演、扮演。
			\subparagraph{verstellen}
			4次。移动、调节;调错;堵住、阻挡;伪装、伪造、变换。
			\subparagraph{herstellen}
			3次。制造、生产;建立、确立;把…放过来;产生、形成。
			\subparagraph{sicherstellen}
			2次。保证、确保;封存、扣留。
			\subparagraph{nachstellen}
			2次。再次校准、再现原貌。
			\subparagraph{abstellen}
			1次。存放、停放;把…暂时搁置一旁;调派;关上;克服、排除、改掉;使适应、针对。
			\paragraph{}			
			最后海德格尔还使用了Konstellation一词4次,其本意包括“形势、局势、情况;星座的位置”。它是否与stellen有关,放在下文再行论述。
			\paragraph{}	
			从名词、动词、分词等词性的角度,对词频最高的3个stellen词族词语进行统计。应注意到,stellen本身源于动词,作动词用有时是用其普通意义,但一旦将其做名词用,尤其是直接从动词形成的名词,则这个词在语句中一定是讨论的一个重点了。
			\paragraph{}	
对于出现28次的stellen,统计如下。动词stellen出现16次,分布于34到62段。动词的名词化形式Stellen出现8次,分布于36到76段。作“地点、处所”之义的名词Stelle出现3次,位于36到39段。
			\paragraph{}	
对于出现共53次的bestellen,统计如下。动词bestellen出现15次,分布于35到53段。动词的名词化形式Bestellen出现21次,广泛分布在35到103段。名词Bestellung出现两次,在36和37段。表能力或可能性的bestellbar,包括Bestellbarkeit, bestellbarem, bestellfähig共出现7次,分布在37到58段,其中有两次以名词形式出现。第一分词bestellende出现4次,分布在43到55段,全部以修饰名词的形容词形式出现,其中3次修饰Entbergen。其他包括Besteller等。
			\paragraph{}	
对于出现18次的vorstellen,统计如下。动词vorstellen出现7次,分布于7到92段。动词的名词形式Vorstellen出现6次,分布于25到72段。名词Vorstellung出现4次,分布在3到12段。			
		\subsubsection{Wesen词族统计分类}
			\paragraph{}
			属于Wesen词族的词语在全文中出现134次。
			\paragraph{}
			从前缀来看,除Wesen本身之外,用了anwesen(在场,出席;存在)及其派生词有14次,其中作动词anwesen仅有1次,位于第39段。动词的名词形式das Anwesen有4次,位于20到25段。第一分词全部以名词出现,das Anwesende或Anwesendes(集合名词)共9次,分布在20到76段。
			\paragraph{}
排除anwesen,Wesen词族在文中出现120次。主要作名词das Wesen有84次,从第1段直到116段。作动词wesen有9次,分布于31到113段。作第一分词时,全部名词化,Wesende有15次,分布于56到116段。
		\subsubsection{Ge-stell统计}
			\paragraph{}
			海德格尔提出的“座架”概念,Ge-stell出现51次。从第48段下定义开始,到第97段。值得注意的是,除了刚提出这一概念是谈到座架原义,使用了5次“Gestell”之外,所有作为技术本质使用的座架都写作“Ge-stell”。
	\subsection{stellen(摆置)词族分析} 
		\subsubsection{stellen词源分析}
			\paragraph{}
		根据词源词典对stellen、Gestell、Bestand、Konstellation的词源解释进行分析,得到以下几个结论。
			\paragraph{}
		dtv的词源词典
		\cite{yellow_etym}对stellen的词义解释是:an einem Ort zum Stehen bringen, aufstellen。对stehen的词义解释是:auf die Füße gestellt sein, auf einer Stelle verharren。可见这里作互解处理(况且Stelle是由stellen衍生的一个名词)。此书将这两个词都最早被追溯到8世纪的古高地德语。再向上溯源,便是两词在8世纪的共同祖先Stall一词,stellen和stehen均由此衍生。
			\paragraph{}
		stellen在发展过程中衍生了大量含义,还与非常多的前缀得以配合,既有动作性,后来又有定位性。其中,bestellen一词的出现是最早的,在9世纪出现,仅比stellen晚了一个世纪。
			\paragraph{}
		Konstellation来自拉丁语,有“stell”的词形,而且表示的也是星之“座”Stellung,但与stellen没有词源联系。Konstellation一词16世纪是天文学用语,后来衍生出星座预言人的命运的含义,现作“形势、局面、状况”。
			\paragraph{}
		为什么海德格尔用stellen而不是之前的fordern来描述技术本质?stellen有与fordern可比的动作性,也能表现位置性,与stehen Bestand 有一定关联,此外它有更丰富的词义、丰富的词族。Gestell有日常作“座架”的形象性,不会像“Gefordern”一般奇怪。stellen这个词能够携带海德格尔厚重的哲学内涵,又能让人联想到星座(Konstellation)。这是fordern做不到的。所以fordern只是作为stellen的引出和形容描述,而stellen才是文中的核心词语。
		\subsubsection{海德格尔对stellen词族的用法}
			\paragraph{}
		开篇(33段之前)用到的stellen词族有vorstellen和feststellen,以及它们的名词形式,还用到了herstellen。vorstellen和Vorstellung用来指称海德格尔给出的一系列通常观念,如把技术视为中性物的观点、两种对技术的通常见解、工具性观点。海德格尔探讨了这些观点的“确定”(Feststellung)性质,把它们从“真实”区分开来。32段再次出现的两个Feststellung与前文的Feststellung含义相同,此处用来描述技术与物理学之间“确定”的交互关系。Herstellung被用来说明工匠制造银盘,在26段还出现了第一分词作形容词形式的“生产制作”(herstellenden),此处是作者要将生产制作也归入解蔽之中。可以说在本文开篇,stellen词族就已经被海德格尔使用了不少次,尽管此时其含义未被被严格规定,但已经为下文埋下伏笔。
			\paragraph{}
		在34到41段,stellen词族被海德格尔大规模集中使用,带来了丰富的语料。这几段文字里,stellen词族中的用法也是多样的,但多数趋于贬义,而且后出现的stellen往往包含了之前出现的stellen的某些含义。这里对该词族的用法包括如下:1) stellen作“提出”,只用在了最开头。表示向自然提出(无理)的要求,这种用法中,stellen的主语和宾语虽然与后文的普遍用法不太一致,但已经给stellen了一个贬义的含义,后文的stellen也会带有一些的“提出无理要求”的含义。2)bestellen作“耕种”,海德格尔解释为“hegen und pflegen”(看护、护理以及照顾、照管等),这是一种没有“促逼”含义的bestellen,海德格尔用anheimgeben(把…托付给,信赖)和hüten(保护,守护)这些十分亲切的词语来进一步描述了这种正面的bestellen。3)herausstellen摆出。4)abstellen适应(对高效大量利用的推动)。5)stellen作建造,水力发电厂被建造在莱茵河上。
			\paragraph{}
		除了上述几个个别用法外,最经常的用法则与“耕种”完全不同。在35到37段中,有多达20个stellen词族的词语,多数以动词形式stellen(摆置)或bestellen(预订)出现,而stellen和bestellen在这里没有内涵上的区别。海德格尔呈现的是一种连锁式的摆置、预订关系。海德格尔举例说明,现代的采矿业、食品工业、工厂、发电厂、旅游工业都处处体现着促逼意义上的摆置、订造的特征。海德格尔呈现的是一种连锁式的摆置、预订关系。在采矿业中,土地为矿石而被摆置,矿石举例来说为铀而被摆置,铀为毁灭或和平的目的而被释放。煤因为其中蕴藏的太阳的热量而被预订,太阳的热量为热能而被预订,热能提供蒸汽,蒸汽为驱动装置提供压力,驱动装置维持工厂运转。这种连锁关系中的“stellen”就是被提供,被放置并准备好,和进一步提供进而提出。
			\paragraph{}
		39段为了提出Bestand一词,海德格尔使用了多次名词形式的Stellen Stelle Bestellen Bestellte。Stelle即wo etw. steht\cite{yellow_etym},由“在位置上”用上了stehen一词,然后用到Stand,并将这种特殊的Stand命名为持存物(Bestand)。“持存物”(Bestand)是海德格尔对被订造的东西的命名。40段以飞机为例,指出Bestand如何处于bestellt之状态。它处处被订造,它的每个零件都是可订造的,它又为下一级的订造随时做好准备,保障(sicherstellen)着运输的可能性。	
			\paragraph{}
		stellen作动词的另一个集中区域是53段谈到自然科学的时候。这里出现了一个在文中多次重复的用语“Natur als einem berechenbaren Kräftezusammenhang(把自然作为一个可计算的力之关联体)”。这个用语在句中的动词可以是可分动词nachstellen,译为把自然加以追踪,也有时是darstellen(53段与73段),将自然呈现出来。物理学摆置自然,实验被订造,是为了探究“如此这般被摆置的自然”是否和如何显露。
			\paragraph{}
		在后面的文字中,不再有stellen词族多次出现的情况了,海德格尔用stellen的集合Ge-stell来阐释后面他所要阐释的东西。
		\subsubsection{《技术的追问》中stellen哲学含义}
			\paragraph{}
			基于上述stellen在《技术的追问》一文中用法的分析,我们可以看到,stellen的词族是丰富的,海德格尔在文中用到的该词族词语也多种多样。我们更深入一步,归纳stellen的哲学含义。我们可以看到stellen的两义性,以及stellen的主体和对象的特点。
			\paragraph{}
			stellen基本的一个含义是正面的。《技术的追问》开篇的一大任务是破除对技术的一般认识,而将技术与解蔽(Entbergen)相关联起来。为此,海德格尔追溯到亚里士多德的四因说。他从希腊语的“原因”中辨读出了“招致”(Verschulden),之后以银盘为例解读“原因”与“招致”。在以银盘为例的论述中,海德格尔从“招致”推及“引发”(Ver-an-lassen)和“产出,生产,创造,带来”(Her-vor-bringen),指出“产出”从遮蔽状态而来,进入无蔽状态中。于是最终海德格尔将技术与“解蔽”紧密联系起来。达到“解蔽”概念后,海德格尔又从Technik的希腊语词源上进行分析。由此发现“技术”一词原初带有“艺术”、“创作”的含义,并与“认识”有很大关联。这些分析进一步证明了技术与解蔽的关系。在这个过程中,用到的stellen词族可以分为两类,一类包括vorstellen和feststellen,它们是与观点、认识等含义相关的。另一类是herstellen,带有产出、创作(Her-vor-bringen)的含义。这两类恰好与“技术”原初的含义相契合。在这个过程中出现stellen词族已经沾上了解蔽的色彩,而且这些stellen词族词语并不带有后来出现的负面含义。stellen的这一片纯净为后面Ge-stell的提出做了准备。在经过了对stellen大量负面的描述之后,52段作者却说道:
			\paragraph{}
			Das Wort stellen meint im Titel Ge-stell nicht nur das Herausfordern, es soll zugleich den Anklang an ein anderes Stellen bewahren, aus dem es abstammt, nämlich an jenes Her- und Dar-stellen...在“座架”这个名称中的“摆置”一词不仅意味着促逼,它同时也保持着与它由之而来的另一种“摆置”的相似,也即与那种制造和呈现的相似。
			\paragraph{}
			这里就与开篇留下的伏笔作了照应。Her-stellen和Dar-stellen正是前文提到的stellen词族两类带有正面含义的词语,也是“技术”一词原初的两方面含义。
			
			
			\paragraph{}
			stellen最明显的一个含义却是负面的。它主要体现在连锁式的摆置与订造链条之中。与stellen负面含义直接相关的词语有两个,分别是“促逼”(herausfordern)和“解蔽”(Entbergen)。《技术的追问》文中大量的语句都在描述stellen和这两个词之间的关系。
			\paragraph{}
			Das in der modernen Technik waltende Entbergen ist ein Herausfordern, das an die Natur das Ansinnen stellt…现代技术中起支配作用的解蔽是“促逼”(Herausfordern),它向自然提出(stellen)蛮横要求。(34段。)
			\paragraph{}
			Das Stellen, das die Naturenergien herausfordert, ist ein Fördern…这种促逼着自然能量的摆置乃是一种开采……(36段。)
			\paragraph{}	
			Das Entbergen, das die moderne Technik durchherrscht, hat den Chatakter des Stellens im Sinne der Herausforderung.贯通并统治着现代技术的解蔽具有促逼意义上的摆置之特征。(38段。)	
			\paragraph{}
			Ge-stell heißt das Versmmelnde jenes Stellens, das den Menschen stellt, d.h. herausfordert, das Wirkliche in der Weise des Bestellens als Bestand zu entbergen.座架意味着对那种摆置的聚集,这种摆置摆置着人,也即促逼着人,使人以订造方式把现实当作持存物来解蔽。(51段。)
			\paragraph{}
			从上面的例句中,可以看到摆置和促逼两词呈现多种关系:促逼摆置自然(这里stellen可以作提要求来讲)、摆置也就是促逼自然、摆置带有促逼的意义。因此可以说负面意义上的“摆置”同“促逼”这个词是形影不离的,有时甚至可以互相替代的。再从例句中看解蔽与它们的关系:这种解蔽是一种促逼,解蔽具有摆置的特征,解蔽是以订造的方式进行的。注意这里的“解蔽”是带限定语的,即“现代技术中统治”的解蔽,解蔽相比摆置和促逼,是一个更广的概念,但如果仅在现代技术这个领域来谈“解蔽”,那么“解蔽”依然可以与“摆置”和“促逼”等同起来,它们的含义都是负面的。
			\paragraph{}
			除了“促逼”和“解蔽”,形容“摆置”的词语还有Steuerung和Sicherung,即“控制”和“保障”是技术的摆置这种解蔽方式的主要特征。


			\paragraph{}
			接下来我们从对象和主体的角度看“摆置”。从42段开始,海德格尔探讨了stellen的主语,即谁实行这种摆置的问题。在这过程中,海德格尔又举了护林人与树林的可订造性的例子,这说明stellen的实施对象既可以是自然,也可以是人。直到45段得出初步的否定式结论,stellen不是纯粹的人的行为,即它的主体并不是人。在多次出现的语句“das Wirkliche (Sichentbergende) als Bestand zu bestellen.”中,海德格尔试图探究动词bestellen的主语如果不是人,那么会是什么。
			\paragraph{}
			紧接着海德格尔由stellen词族在48段引出了Ge-stell(座架)这个著名的概念。在51段海德格尔终于在句子中明确了主语:Ge-stell heißt das Versmmelnde jenes Stellens, das den Menschen stellt, d.h. herausfordert, das Wirkliche in der Weise des Bestellens als Bestand zu entbergen.做出stellen动作的这个主语就是“座架”(Ge-stell)。
			
			
%			后文中海德格尔也解释了Gestell词干中体现的stellen之含义。首先,前缀“Ge-”意味着它是stellen的集合(51段),是stellen词族中各个词语含义的集合,其次这里的词干保持着与它由之而来的另一种Stellen的相似,即Her- und Dar-stellen(产出和呈现)(52段)。它们让海德格尔联系到了“创作”这个希腊词语,进而联系到了无蔽,在Ge-stell中发生(sich ereignen)着无蔽状态。	
		
		
		
%	\subsection{中、英译文解读}
%		\paragraph{}
%		正如前文所述,在《技术的追问》中,stellen的含义非常丰富,词族庞大,有动作性,也能表现位置性,与stehen Bestand 有一定关联。在其他的语言中是不太可能找到能够代替的一个词语的,用其他语言中多个相互关系不大的词语翻译stellen词族中词义不同却关联紧密的各个词语,必然使得表意大打折扣。但即便如此,借鉴中、英文翻译者堆某些词语的翻译方式依然能为加深理解提供帮助。
%		\paragraph{}
%		William Lovitt的英译本《The Question Concerning Technology and Other Essays》\cite{Quest_Concern_Tech}是《技术的追问》较为权威的英译版本。英译文将stellen翻译为to set或to set upon,有时被译为to supply,将bestellen译为order或on call。但这些词语远不能表达stellen的完整含义,因此译者写了详细的译注:
%		\paragraph{}
%		The verb stellen (to place or set) has a wide variety of uses. It can mean to put in place, to order, to arrange, to furnish or supply, and, in a
%military context, to challenge or engage.\cite[pg. 15]{Quest_Concern_Tech}
%		\paragraph{}
%		对于Ge-stell,Lovitt译为Enframing,突出了Gestell的支架原义,而不刻意与stellen(set on, place)相呼应。
%		\paragraph{}
%		国内对《技术的追问》的翻译主要有孙周兴教授,事实上他的译本除了依据海德格尔原文外,还重点参考了Lovitt的英译本。
%		\paragraph{}
%		在孙周兴的文章《学术翻译的几个原则》\cite{Aca_Transl_SZx}中,孙周兴提出了语境原则、硬译原则、统一原则、可读原则这四项翻译方针。这其中,重点是硬译,即“宁取字面义,勿取解释义”,他还说道:哲学-思想类的译文就要蓄意地做得硬梆梆的,让一般的人们看不懂——因为原著本身就不是一般人所能接近的。若是把学术品也译得喜闻乐见……则学术翻译的意义已经丧失了大半。
%		\paragraph{}
%		以硬译为原则,孙周兴将stellen译为“摆置”,将bestellen译为“订造”,将Ge-stell译为“座架”。有些学者将stellen译为“拷打”,将Ge-stell译为“集置”,孙周兴与之形成鲜明对比。孙周兴的译法不直接揭示词语含义,是考虑到几乎无法找到能表达stellen或Ge-stell多方面含义的中文词语,与其用片面、夸张、生造的词语,不如硬译,让其哲学含义在行文中被揭示。
		
	\subsection{Wesen(本质)及相关词语分析}
		\subsubsection{Wesen相关词语的词源}
			\paragraph{}
		《技术的追问》文中对本质一词已经进行了词源的探究,但是他的某些步骤将一词与另一词联系活不够严谨,或只是一笔带过。因此这里研究Wesen相关词语的词源,将囊括wesen、essensia、währen、Wahrheit、gewähren等词,将起到补充、印证或质疑海德格尔原文的作用。
			\paragraph{}
			\subparagraph{wesen}
			该词可追溯到8世纪的1古高地德语的wesan,在那时表示“在某地的停留(Verweilen),留存、存在、生存(Dasein)”。该词名词形式来自不定式动词,它属于sein范畴下的强变化动词。后来还发展出了居所、物、状态等含义。
			\subparagraph{währen}意为dauern、Bestand haben、 verweilen即持续,同时有存在之义。während作为弱变化持续体动词,是由名词Wesen对应的强变化动词wesen形成的。而且从历史词义上看,它们也有共同之处。
			\subparagraph{wahr, Wahrheit, wahren}
			währen的词形很容易让人联想到wahr(真的)、Wahrheit(真理、真相)和wahren(保护维护、维持)。在词源词典中,wahr和wahren是分开的,而Wahrheit归属于wahr的子条目。
wahr在古高地德语中就有“真(wirklich)”“确凿(gewiß)”的意思。Wahrheit属于wahr的派生词,在8或9世纪就有warheit。在wahr词条中,还提到了gewähren。wahren在古高地德语中表示结盟的忠实、保护、契约、保障等含义。词源词典中把wahren与wahr分开为两个词条进行解释,但这两个词的关系较为暧昧,在古高地德语、中古高地德语中古低地德语等中有大量词形非常相似的拼写方式。wahren与古高地德语的wahr有亲缘关系,不过在任何时期内wahr与wahren都没有等同过。但不论如何,wahr与wahren与währen从词源上没有直接联系
			\subparagraph{gewähren}
			该词在9世纪的古高地德语中就已经独立出现,意为Gewähr leisten, zugestehen,而Gewähr就是保障、担保之意,与wahr在古高地德语中的意思相近。gewähren与währen反倒没有什么关系。
			\subparagraph{essentia}
			德文作die Essenz,直译为Wesen,Wesenheit,词义为“精华、核心”,也有化学的浓缩物之义。essentia是拉丁文,意为“本质”,它的来源是esse,即是,存在,译为sein。可以看到,拉丁文和德文在“存在”和“本质”这里有巨大的相似性,都是从最基本的“是动词”产生出本质,即海德格尔所引用的“所是的那个什么”。
		\subsubsection{海德格尔对Wesen的用法}
			\paragraph{}
		海德格尔对Wesen的使用在全文中可谓从始至终。主要可以分为两大部分,以83段为分界。之前一直以通俗(geläufig)的含义上使用本质一词,而后对本质一词进行真正的探索。
			\paragraph{}
		海德格尔对本质一词的初步解释在开篇就给出了。一方面,海德格尔通过树的例子解释本质,他用了两个词,贯穿支配(durch walten)。贯穿描述了一种共性,即每棵树都被本质所支配,而支配指出,本质不是一颗树,也不是树的某些共性特征,而是高于树的一个抽象概念,它使得这一棵东西成为树(这种抽象对后面的“座架也同样适用”)。另一方面依据一种古老的学说(nach alter Lehre),本质被看作某物所是的那个“什么”(was etwas ist)。这些便是本质的“通俗”含义。
			\paragraph{}
		即使是在通俗的含义上使用本质一词,海德格尔在前文中的这个词也有多种不同用法。从本质的主语进行分析,最主要的当然就是技术(Wesen der Technik),第59、68、70段明显地重复着一句话:技术的本质居于(beruhen)座架之中。凡是说到技术的本质,海德格尔讲的都是座架,而暂不涉及本质这个词本身,比如78段,技术之本质作为解蔽之命运乃是危险。这里说“本质乃是危险”实际是说座架的危险性质。
			\paragraph{}
		除了以技术作为本质的主语,海德格尔还在文中以历史、自由、人做为本质的主语。而这三个主语直接的关系体现在海德格尔的推演过程中。技术的本质将人带上一条路,它将现实(Wirkliche)转变为Bestand。带上路这种遣送(Schicken)是命运(Geschick),而命运规定着的就是历史(Geschichte)的本质。(63段)。紧接着海德格尔谈及自由,人归属于命运领域,成为倾听者,才是自由的,自由的本质不是意志,自由掌管着开放领域,这就联系到了解蔽概念。通过这一系列的推演,海德格尔得以拓展技术的本质中的“解蔽”内涵,探究这是如何的一种解蔽。之后海德格尔用到了“人”作为本质的主语,因为技术的本质指点的解蔽是一种stellen的单一形式的解蔽,一种贬义的解蔽,解蔽技术性的同时,人的本质却被遮蔽了,人不再碰到自身。这便是危险,座架的危险在人的本质处触动了人类。
			\paragraph{}
		海德格尔还用到了3次本质的动词形式wesen。其中两次与技术或座架有关,一次是在介绍柏拉图中使用。31段提到:技术在真理发生的领域现身(west)。62段提到,我们是否和如何真正投入到座架本身现身(west)于其中的那个东西中。这两个地方,包括介绍柏拉图的一个地方,wesen都与一个介词in共同出现,wesen都发生在一个领域之中,某物在一个领域中成其本质。在这些上下文中,如果把wesen改成sein,句子也可以说得通,从词源上我们也有依据,允许我们如此理解。
			\paragraph{}
		从对本质一词的描述来看,海德格尔始终说要揭开(enthüllen, ans Licht zu heben)本质,但从历史上看,本质却如纱巾(Schleier)遮蔽(verhüllen, verbergen)着自身。因此57段说,现代技术虽然晚于自然科学的发展,但技术的本质则是历史上(geschichtlich)早先的东西。
			\paragraph{}
		
		\subsubsection{Wesen相关词语的哲学含义与关联性}
			\paragraph{}
		到了83段以后,Wesen一词被海德格尔重点探究。首先海德格尔否定了对Wesen的一些认识。84段重复了开头第4段的定义,从哲学的学院语言中,本质的意思是was etwas ist,某物所是的那个“什么”,即Washeit。接下来海德格尔由之前的通俗或学院语言中的本质而上升了一个高度,并再次否定了本质的种类之说。海德格尔使用Weise一词解释技术的座架本质Das Ge-stell ist eine geschickhafte Weise(方式方法) des Entbergens, nämlich das herausfordernde… Aber diese Weisen sind nicht Arten… (84段)。85段得出否定性结论,座架不是种类(Gattung)或essentia意义上的技术本质,这里的essentia的历史由来在本论文前面的词源研究中提及,essentia意义上的本质侧重于拉丁文词干esse,即“是,存在”,与上述“学院语言”的本质含义相一致,而海德格尔强调的却不是这种本质。接下来海德格尔开始从Weise出发,从另一种意义上去解释“本质”。
			\paragraph{}
		接下来海德格尔从语用和词源角度分析了Wesen一词。延续上文提出的Weise,举例“家政”“国体”来说明,用walten, sich verwalten, entfalten, verfallen几个词进一步解释了Weise。这里的这些动词在句中位置如下:die Weise, wie Haus und Staat walten… 而后面一句用了完全相同的句式:Es ist die Weise, wie sie wesen.海德格尔用上述的四个词来解释了动词wesen。由动词发展出名词Wesen,而后海德格尔从词源的角度将Wesen与währen(持续)相联系这一点的正确性在本论文的词源分析中得到证明。又由苏格拉底和柏拉图的思考,印证了Wesende带有的Währende的意义,并且他们的Währende是作为Fortwährende(永久持续者)来进行思考的,而对Fortwährende的寻找是在那些作为Bleibende(留存者)坚持于一切出现之物的东西中进行的。
			\paragraph{}
		从fortwähren海德格尔继续他的探索。fortwähren这个词让“本质”显得过于观念化、抽象化。海德格尔引出歌德的作品中的一个“神秘”的词语fortgewähren,并说道:“他的耳朵在此听出了währen(持续)和gewähren(允诺)两词之间的未曾道出的一致。”然而,海德格尔没有详细给出fortgewähren在歌德作品里是如何被使用的。从词源的角度看,währen和gewähren也无法扯上关系。
			\paragraph{}
		fortgewähren一词来自短篇小说《一对邻人儿女的奇缘(Die wunderlichen Nachbarskinder)》(孙周兴海德格尔选集中对小说的题目翻译有所错误)。小说用到该词的段落是阶段性总结的一段。描述的是一对青年男女的处于准备结婚的状态(但后来女方遇到小时候的男伴后故事发生转折)。这段的原文如下:
			\paragraph{}
		“Der ruhige Gang, den die ganze Sache genommen hatte, war auch durch das Verlöbnis nicht beschleunigt worden. Man ließ eben von beiden Seiten alles so fortgewähren, man freute sich des Zusammenlebens und wollte die gute Jahreszeit durchaus noch als einen Frühling des künftigen ernsteren Lebens genießen.”\cite{wNachbarK_Goet}
			\paragraph{}
		本段的中文翻译如下:\cite{wNachbarK_trans}
			\paragraph{}
		他们的事情进展平稳,即使是通过订婚也没加快事情的进程。双方都继续听其自然。他们愉快地相处在一起,都心安理得地把这一段美好的时光当成未来较为严肃的婚姻生活的春天来尽情享受。
			\paragraph{}
		译者将fortgewähren译为“听其自然”,明显不是直译或“硬译”,毕竟文学作品的翻译和学术文章的翻译是不同的。这里的fortgewähren一词,如果换用fortwähren也完全说得通,即:双方都得以(将共同生活)持续下去。事实上,fort-有继续之义,而währen的意思也是持续、延续(同时有存在之义)。而使用fortgewähren事实上是借用了一个gewähren的固定句型,jemanden gewähren lassen(听某人自便,对某人的行为不再过问或干预)。歌德的这句保留了持续之义的同时,增加了人们(man)对这对未婚夫妻的允诺、准许之义。同时给出了海德格尔未提及的“允诺”一词的主语“man”,但这个主语依然没有明确到底是谁来允诺。
			\paragraph{}
		从对歌德的引用,海德格尔得到一个重要结论:Nur das Gewährte währt. Das anfänglich aus der Frühe Währende ist das Gewährende.\cite{Frage_n_Tech}孙周兴将这两句译为:只有允诺者才持续。原初地从早先而来的持续者乃是允诺者。\cite[pg. 949]{slct_Hei_Szx}从“持续”借歌德联系到“允诺”,并定下“只有允诺者才持续”的结论,这里不得不说是海德格尔的思路中少有的跳跃。海德格尔的这两句话引出了后面针对“座架”的允诺救渡特性的问题和解释。座架作为技术的本质现身(das Wesende),乃是持续者(das Währende),那么这一持续者是在允诺者意义上(im Sinne des Gewährenden)运作(walten)的吗?\cite[pg. 949]{slct_Hei_Szx}乍看起来,座架的促逼不是允诺,但无论如何,促逼总是给人指点一条解蔽道路的遣送(Schicken)。因此,即使这种座架的Schicken中蕴藏极端的危险,它也依然是允诺。而允诺者就是救渡(Rettende),人对允诺者的归属性在危险之中恰恰显现了出来。
			\paragraph{}
		至此,海德格尔先以词源为线索,从本质一词一步步推到允诺一词,然后阐明座架作为本质的允诺特性,并得到座架将人遣送入解蔽的救渡特性。
			\paragraph{}
		这时我们回到81段对荷尔德林诗句的阐释,这里事实上预先解释了救渡和本质的关系(尽管还没有提出“允诺”这个词)。“救”乃是:把……收取入本质中,把本质带向真正的显现。这样,本质、允诺、救渡三个词就呈现出两两相关的特性了。
	\subsection{座架(Ge-stell)}
		\subsubsection{座架词源}
		\subsubsection{座架概念的提出}
		\subsubsection{座架作为技术的本质}
		\paragraph{}
		自座架(Ge-stell)由对stellen的主语的追问而被提出之后,大部分的文字都是围绕座架的。那么座架在何种意义上是技术的本质呢?在贯穿全文的论述中,座架作为技术的本质,其本身不是技术性的东西 ,事实上座架始终是一个矛盾的统一体。
		\paragraph{}
		座架是各种stellen的集合。在这个集合里,海德格尔提到了stellen词族的两类词语:有摆置(stellen)与订造(bestellen),也有展现(darstellen)、生产(herstellen)。前一类带有明显的贬义,这些可谓一种促逼(herausfordern),而后一类带有褒义,它们提醒着我们,座架也是一种解蔽的方式(Weise des Entbergens),在座架之中发生着无蔽状态(Unverborgenheit)。因此,座架是促逼和解蔽方式的矛盾统一体,它以促逼的方式将现实作为持存物来解蔽。
		\paragraph{}
		进一步来说,上述这种促逼的解蔽方式像每一种解蔽方式一样,都给人指点道路,将人遣送入解蔽途中。在德文中,遣送(schicken)、历史(Geschichte)和命运(Geschick)同族,词根都有遣送之意。于是座架就带有了解蔽之命运遣送的意义。从另一个方面说,人们被座架遣送入一种促逼方式的解蔽途中,这就遮蔽(verbergen)、驱除(vertreiben)了艺术、神性等其他可能性,这意味着座架带有着最高的危险(die höchste Gefahr)。正如78段总结的那样,座架的意义包含了命运和危险,座架是命运遣送和危险的矛盾统一体。
		\paragraph{}
		最后,结合着座架的命运遣送意义,从荷尔德林的诗句、本质与允诺的词语关联等方面,海德格尔找到了座架中孕育的救渡(Retten)。海德格尔最终对座架的总结正是从座架的两义性(Zweideutigkeit)来说的:一方面,座架促逼入那种订造的疯狂中,危害着与真理之本质的关联;另一方面,座架发生于允诺者中,如此便显现出救渡之升起。
%	\subsection{中、英译文解读}
%		\paragraph{}
%		在中文翻译中,孙周兴使用“本质”来翻译Wesen,这没有什么问题,而本质的动词形式wesen译为成其本质(31段,94段),或现身(50段讲柏拉图,62段讲座架,86段解释本质的动词由来,89段技术“持续物”),类似的,Wesende译为现身之物,或本质现身之物(89、90段讲到永久持续和允诺,以及92段),最后讲艺术的durchwest(111段)译为贯通,而不再带有本质的意思了。
%		\paragraph{}
%		英文翻译中,本质的动词译为come to presence(31段,62段)endures as present(50段),86段直接加引号用名词翻译动词,再用come to presence解释,89段也用名词翻译动词。第一分词名词形式除了第一次用“what essences”(86段),后面用了动名词形式essencing(89,90)来翻译Wesende,还用了coming to presence,亦是动名词形式(90)。名词的译法则有时使用essence有时使用coming to presence,这是考虑到海德格尔说道的85段:Wesen不是essentia意义上的。
\section{结语}
	\subsection{总结}
		\paragraph{}
		本论文从stellen和wesen两个词语的角度开始对海德格尔的《技术的追问》展开研究。stellen被海德格尔用来描述技术对待自然和人的方式,其词族含义丰富,能够容纳技术的“促逼”与“解蔽”两方面特性,因此被选用来揭示技术摆置与订造的疯狂。探究stellen主语得到“座架”概念,并将“座架”作为技术的本质。海德格尔先以wesen的通行含义探究“座架”的促逼和危险、解蔽和遣送,后来将本质联系到“方式”,从词源联系到“持续”,从歌德作品联系到“允诺”,为本质发展出新含义,最终找到座架中蕴藏的救渡。
	\subsection{展望}
		\paragraph{}
		鉴于论文撰写时间有限,《技术的追问》中更多的体现海德格尔作文特色的词语溯源和联系便无法一一分析了。比如《技术的追问》中对hören词族中的gehören、Hörender、Höriger的使用,以及对schicken词族的geschichtlich、Geschehen、Geschichte的使用,以及其与historisch的区分。这些地方处处体现着海德格尔的思与智慧,值得将来进一步分析研究。除此之外,表达海德格尔技术思想的文章不止《技术的追问》,还包括《转向》《世界图像的时代》等等,对这些文章的系统性分析将带来对海德格尔技术思想更完整、更丰富的认识。
	\subsection{技术的座架本质对技术时代的启示}
		\paragraph{}
		
\renewcommand\refname{参考文献}
\bibliographystyle{plain}
\bibliography{citations.bib}
\end{document} 