\documentclass{article} 
\usepackage{xeCJK}
\usepackage{cite}
\setCJKmainfont{Droid Sans Fallback}
\author{叶欣} 

\title{《技术的追问》词语分析} 
\begin{document}   
\maketitle   



\begin{abstract}

\end{abstract}



\tableofcontents

\section{引言} 

	\subsection{选题的意义} 
		\paragraph{}
海德格尔是20世纪最伟大的思想家之一,他对技术问题的探究是他后期思想的重要组成部分。针对技术本质的问题,海德格尔提出了座架(Gestell)这一概念,并在海德格尔哲学的语境下进行了深入探索。由于海德格尔语言特点略显晦涩,其写作中“座架”“本质”等概念较为抽象,有时词义被海德格尔拓展,有时词语被海德格尔溯源,并与其他词语相关联,往往令人难以理解。因此本研究认为,从分析德文词语词源、基本词义、词族关联等方面入手,是理解海德格尔技术之思的重要途径。
		\paragraph{}
海德格尔哲学,尤其是他关于技术的论述引发了中国学界的重视。在译介方面,以孙周兴为代表的学者翻译了海德格尔大量演讲与论文,对海德格尔思想表述中Ereignis、Dasein、Gestell等重点概念的含义和译法进行了深入探讨。在对海德格尔技术思想的解读方面,国内已有大量论著,如范玉刚的《睿思与歧误:一种对海德格尔技术之思的审美解读》,包国光的《海德格尔生存论视域下的技术》,研究论文更是多如牛毛。但是,从语言、词语的角度出发,对海德格尔技术思想重点概念所进行的研究却尚未看到,这不得不说是一个遗憾。
		\paragraph{}
此外,本研究尝试结合中、英文翻译用词进行研究,从词语角度辨析多种语言对海德格尔哲学的各自解读,能极大帮助我们理解德文词语的完整含意,并认识不同文化对海德格尔哲学的引介情况。
	\subsection{问题的提出}
		\paragraph{}
     	从词语角度出发,我们提出以下问题。
		\subparagraph{1} 
            在海德格尔思想中,为何用stellen(摆置)解读现代技术?
		\subparagraph{2} 
           “技术的本质”中“本质”(Wesen)的含义是什么?
		\subparagraph{3}海德格尔为何提出Gestell(座架)一词?它在何种意义上是技术的本质?
		\subparagraph{4}中文、英文对德文词语的翻译能否为我们理解海德格尔提供线索?
		\subparagraph{5}技术的座架本质对身处技术时代的我们有何启示?
	\subsection{研究目的}
		\paragraph{}
        本课题拟通过对海德格尔《技术的追问》中核心词汇的深入解读,通过对Gestell(座架)和Wesen(本质)等词汇的词源、所属词族、中英文翻译的研究,结合海德格尔相关文章的用词背景、在本文中对词语关系的推演,以达到认识德语词语的完整含义、追踪海德格尔技术之思的路途、深入理解海德格尔所谓“技术本质”的目的,并为技术时代的生活提供借鉴。
	\subsection{论文结构}
		\paragraph{}
        论文的第一部分为引言,介绍本课题的研究意义、目的,提出待研究的问题。
		\paragraph{}
        论文的第二部分,简要介绍海德格尔哲学思想、语言特点,对本课题研究对象《技术的追问》进行总体介绍,并对《技术的追问》中的重点词语及所属词族进行统计和分类,便于后续深入研究的开展。
        \paragraph{}
        论文的第三部分,拟分析stellen(摆置)这一核心词族的词源、用法和含义。理解海德格尔对stellen的论述,认识Gestell(座架)概念如何被提出。结合中英译文的处理方式进行理解。
        \paragraph{}
        论文的第四部分,拟分析“座架作为技术之本质(Wesen)”中,本质一词的词源、用法和含义。区分技术本质与技术性的东西。探索währen、gewähren、Wahrheit等词语是如何从德语角度和海德格尔的角度与Wesen联系起来的。揭示座架何以是技术之本质。
        \paragraph{}
        论文的第五部分为结语,对本文进行简要总结。还会提及海德格尔对其他词语的运用,作为对未来研究的展望。最后,反思技术的座架本质对技术时代的启示。
\section{海德格尔与《技术的追问》} 
	\subsection{海德格尔的技术思想}
		\paragraph{}
	\subsection{海德格尔的语言特点}
		\paragraph{}
		从海德格尔的多篇著述、各国学者对海德格尔作品的翻译、他人对海德格尔的评价来看,可以发现海德格尔的语言有其鲜明特点,主要包括:方向性强、严整性弱、重视词语的使用等。
		\paragraph{}
		海德格尔作品的方向性、顺序性强。其行文过程如同一次探索的经历。正如海德格尔在《技术的追问》开篇所说,“追问构筑一条道路。因此之故,我们大有必要首先关注一下道路,而不要牵挂于个别的句子和明目。”
\cite{slct_Hei_Szx}
基于这样一个基调,在海德格尔行文过程中,一个词语可以在文中多次出现,但后来出现时往往比之前含义更加丰富。比如在《技术的追问》中,本质一词开头就多次出现,但这时“技术本质”的含义只是以“某物所是的那个什么”这个程度上理解的,直到文章中后部,出现一句“直到现在,我们还是在流俗的含义上来理解‘本质’一词的”
\cite{slct_Hei_Szx}
,从这开始,本质一词的含义才被继续开发,并联系到“持续”“允诺”等词语。
在海德格尔的全集前言草稿中,他写道:“Wege – nicht Werke”(道路——而非著作)。在对此的解释中,他说他的全集是“在对多义的存在问题所作的变动不居的追问道路之野上的一种行进(Unter-wegs,在途中)”。
\cite{sprach_sein_Szx}
还可以注意到,道路一词使用了复数形式,这正印证了他的追问的“变动不居”,即海德格尔不认为他的思想会像古典哲学那样搭建一座形而上学的完整建筑,而是从多个方向探索,行进在思想道路上。这与海德格尔的教学习惯有密切联系,他不希望学生们只是听从他的观点而已。他希望学生或读者在思之旅途上伴随他,引领学生修筑自己的思想道路。
		\paragraph{}
		正是由于海德格尔语言如同道路一般的特点,他的文章有时令人觉得不够严谨。伽达默尔说,“后期海德格尔自己为了逃避形而上学的语言而发展出他半诗性的特殊语言”
\cite{Text_Explain_DeFr}
,他甚至用“女巫式风格”来形容其后期著作
\cite[pg. 113]{Deonstr_DeFr}
。这一点在《技术的追问》中也有所体现。比如从währen(持续)到gewähren(允诺),海德格尔仅举歌德作品中一例用词,说歌德的“耳朵在此听出了两词之间的未曾道出的一致”,便下了结论“只有允诺者才持续”
。又比如他所说危险中的解救,是以“荷尔德林的诗句道出了真理”为前提的\cite{slct_Hei_Szx}。
海德格尔有时也以比喻代替论证,比如在《技术的追问》结尾使用的Konstellation(星座)一词。
		\paragraph{}
		海德格尔说过,“语言是存在之家”,对词语的重视是海德格尔思想的重要特点。在行文过程中,他有时探究词源、有时阐释词语日常用法、有时赋予词语新含义。如伽达默尔所说,海德格尔甚至喜欢把词语引回到它们已经失落了的不再具有的词义上去,并从这种所谓词源学的词义中得出结论
\cite{Deonstr_DeFr}。溯源的词语多为希腊文或德文,古希腊文富于哲学内涵,故德文相对纯净朴素。在《技术的追问》中海德格尔就先后对两个核心词语Technik和Wesen分别进行了希腊文和德文的词源探寻。探究一个词族中词语的关联就不胜枚举了,比如stellen schicken hören wesen等词的词族。给词语赋予新含义的例子也很多,比如前期哲学中描述人的存在状态的烦(Sorge)、畏(Angst)、《技术的追问》中的座架(Ge-stell)等等。
	\subsection{《技术的追问》简介}
		\paragraph{}
		《技术的追问》来源于海德格尔的演讲。1949年12月1日,海德格尔在不莱梅进行了四个演讲,题目分别为“物(Das Ding)”“座架(Das Gestell)”“危险(Die Gefahr)”“转向(Die Kehre)”。1954到1955年期间上述第二个演讲被扩充,以“技术的追问(Die Frage nach der Technik)”为题,作为演讲系列“技术时代的艺术”的一部分。1962年“技术的追问”被收入《技术与转向》一书。
	\subsection{《技术的追问》词语统计与分类}
		\subsubsection{stellen词族统计分类}
			\paragraph{}
			stellen词族作为《技术的追问》的核心词语,在本文中出现138次(不计Ge-stell)。合并名词动词等词性的区分,仅从词语前缀与词义上进行区分,得到了以下数据。这里附上《朗氏德汉双解大词典》中对这些词的中文解释,这些词的某些含义海德格尔在《技术的追问》中并未使用,但它们的各种含义可以让我们对stellen这个词族的含义丰富性有一个大体了解。
			\subparagraph{stellen}
			28次。摆放、放置;拨置,调节;给…配置、提供;提出。
			\subparagraph{bestellen}
			53次。订货、订购;预订;预约、约请;耕种。
			\subparagraph{vorstellen}
			18次。向…介绍;推出、展示;设想。
			\subparagraph{feststellen}
			6次。查明、确定;看出;明确指出。
			\subparagraph{darstellen}
			5次。展示、再现;阐述、描述、说明;描绘、表现;意味着、表示;表演、扮演。
			\subparagraph{verstellen}
			4次。移动、调节;调错;堵住、阻挡;伪装、伪造、变换。
			\subparagraph{herstellen}
			3次。制造、生产;建立、确立;把…放过来;产生、形成。
			\subparagraph{sicherstellen}
			2次。保证、确保;封存、扣留。
			\subparagraph{nachstellen}
			2次。再次校准、再现原貌。
			\subparagraph{abstellen}
			1次。存放、停放;把…暂时搁置一旁;调派;关上;克服、排除、改掉;使适应、针对。
			\paragraph{}			
			最后海德格尔还使用了Konstellation一词4次,其本意包括“形势、局势、情况;星座的位置”。它是否与stellen有关,放在下文再行论述。
			\paragraph{}	
			从名词、动词、分词等词性的角度,对词频最高的3个stellen词族词语进行统计。应注意到,stellen本身源于动词,作动词用有时是用其普通意义,但一旦将其做名词用,尤其是直接从动词形成的名词,则这个词在语句中一定是讨论的一个重点了。
			\paragraph{}	
对于出现28次的stellen,统计如下。动词stellen出现16次,分布于34到62段。动词的名词化形式Stellen出现8次,分布于36到76段。作“地点、处所”之义的名词Stelle出现3次,位于36到39段。
			\paragraph{}	
对于出现共53次的bestellen,统计如下。动词bestellen出现15次,分布于35到53段。动词的名词化形式Bestellen出现21次,广泛分布在35到103段。名词Bestellung出现两次,在36和37段。表能力或可能性的bestellbar,包括Bestellbarkeit, bestellbarem, bestellfähig共出现7次,分布在37到58段,其中有两次以名词形式出现。第一分词bestellende出现4次,分布在43到55段,全部以修饰名词的形容词形式出现,其中3次修饰Entbergen。其他包括Besteller等。
			\paragraph{}	
对于出现18次的vorstellen,统计如下。动词vorstellen出现7次,分布于7到92段。动词的名词形式Vorstellen出现6次,分布于25到72段。名词Vorstellung出现4次,分布在3到12段。
			\subparagraph{Ge-stell}	
			海德格尔提出的“座架”概念,Ge-stell出现51次。从第48段下定义开始,到第97段。值得注意的是,除了刚提出这一概念是谈到座架原义,使用了5次“Gestell”之外,所有作为技术本质使用的座架都写作“Ge-stell”。
		\subsubsection{Wesen词族统计分类}
			\paragraph{}
			属于Wesen词族的词语在全文中出现134次。
			\paragraph{}
			从前缀来看,除Wesen本身之外,用了anwesen(在场,出席;存在)及其派生词有14次,其中作动词anwesen仅有1次,位于第39段。动词的名词形式das Anwesen有4次,位于20到25段。第一分词全部以名词出现,das Anwesende或Anwesendes(集合名词)共9次,分布在20到76段。
			\paragraph{}
排除anwesen,Wesen词族在文中出现120次。主要作名词das Wesen有84次,从第1段直到116段。作动词wesen有9次,分布于31到113段。作第一分词时,全部名词化,Wesende有15次,分布于56到116段。
\section{stellen(摆置)词族分析} 
	\subsection{stellen词源分析}
		\paragraph{}
		根据词源词典对stellen、Gestell、Bestand、Konstellation的词源解释进行分析,得到以下几个结论。
		\paragraph{}
		dtv的词源词典
		\cite{yellow_etym}对stellen的词义解释是:an einem Ort zum Stehen bringen, aufstellen。对stehen的词义解释是:auf die Füße gestellt sein, auf einer Stelle verharren。可见这里作互解处理(况且Stelle是由stellen衍生的一个名词)。此书将这两个词都最早被追溯到8世纪的古高地德语。再向上溯源,便是两词在8世纪的共同祖先Stall一词,stellen和stehen均由此衍生。
		\paragraph{}
		stellen在发展过程中衍生了大量含义,还与非常多的前缀得以配合,既有动作性,后来又有定位性。其中,bestellen一词的出现是最早的,在9世纪出现,仅比stellen晚了一个世纪。
		\paragraph{}
		Konstellation来自拉丁语,有“stell”的词形,而且表示的也是星之“座”Stellung,但与stellen没有词源联系。Konstellation一词16世纪是天文学用语,后来衍生出星座预言人的命运的含义,现作“形势、局面、状况”。
		\paragraph{}
		为什么海德格尔用stellen而不是之前的fordern来描述技术本质?stellen有与fordern可比的动作性,也能表现位置性,与stehen Bestand 有一定关联,此外它有更丰富的词义、丰富的词族。Gestell有日常作“座架”的形象性,不会像“Gefordern”一般奇怪。stellen这个词能够携带海德格尔厚重的哲学内涵,又能让人联想到星座(Konstellation)。这是fordern做不到的。所以fordern只是作为stellen的引出和形容描述,而stellen才是文中的核心词语。
	\subsection{海德格尔对stellen词族的用法}
		\paragraph{}
		开篇(25段之前)用到的stellen词族有vorstellen和feststellen,以及它们的名词形式,还用到了一次Herstellung。vorstellen和Vorstellung用来指称海德格尔给出的一系列通常观念,如把技术视为中性物的观点、两种对技术的通常见解、工具性观点。海德格尔探讨了这些观点的“确定”(Feststellung)性质,把它们从“真实”区分开来。出现了一次的Herstellung被用来说明工匠制造银盘,但并未对“制造”从词语的角度作深入探讨。可以说在本文开篇,stellen词族是在非常平凡的意义上被海德格尔使用的,此时的stellen并不涉及到其技术哲学思想的核心概念。
		\paragraph{}
		从26到33段,stellen词族用得很少。在26段除了与前面意义相同的vorstellen之外,还出现了第一分词作形容词形式的“生产制作”(herstellenden),此处是要将生产制作也归入解蔽之中。32段再次出现的两个Feststellung与前文的Feststellung含义相同,此处用来描述技术与物理学之间“确定”的交互关系。
		\paragraph{}
		在34到41段,stellen词族被海德格尔大规模集中使用,带来了丰富的语料。这几段文字里,stellen词族中的用法也是多样的,但多数趋于贬义,而且后出现的stellen往往包含了之前出现的stellen的某些含义。这里对该词族的用法包括如下:1) stellen作“提出”,只用在了最开头。表示向自然提出(无理)的要求,这种用法中,stellen的主语和宾语虽然与后文的普遍用法不太一致,但已经给stellen了一个贬义的含义,后文的stellen也会带有一些的“提出无理要求”的含义。2)bestellen作“耕种”,海德格尔解释为“hegen und pflegen”(看护、护理以及照顾、照管等),这是一种没有“促逼”含义的bestellen,海德格尔用anheimgeben(把…托付给,信赖)和hüten(保护,守护)这些十分亲切的词语来进一步描述了这种正面的bestellen。3)herausstellen摆出。4)abstellen适应(对高效大量利用的推动)。5)stellen作建造,水力发电厂被建造在莱茵河上。
		\paragraph{}
		除了上述几个个别用法外,最经常的用法则与“耕种”完全不同。在35到37段中,有多达20个stellen词族的词语,多数以动词形式stellen(摆置)或bestellen(预订)出现,而stellen和bestellen在这里没有内涵上的区别。海德格尔呈现的是一种连锁式的摆置、预订关系。

		\paragraph{}
		39段为了提出Bestand一词,海德格尔使用了多次名词形式的Stellen Stelle Bestellen Bestellte。Stelle即wo etw. steht\cite{yellow_etym},由“在位置上”用上了stehen一词,然后用到Stand,并将这种特殊的Stand命名为Bestand。40段以飞机为例,指出Bestand处于bestellt之状态,保障(sicherstellen)着运输的可能性,并且每个零件都是可订造的。
		
		
		\paragraph{}
		stellen作动词的另一个集中区域是53段谈到自然科学的时候。这里出现了一个在文中多次重复的用语Natur als einem berechenbaren Kräftezusammenhang(把自然作为一个可计算的力之关联体)。这个用语在句中的动词可以是可分动词nachstellen,译为把自然加以追踪,也有时是darstellen(53段与73段),将自然呈现出来。物理学摆置自然,实验被订造,是为了探究“如此这般被摆置的自然”是否和如何显露。
		\paragraph{}
		在后面的文字中,不再有stellen词族多次出现的情况了,海德格尔用stellen的集合Ge-stell来阐释后面他所要阐释的东西。
	\subsection{《技术的追问》中stellen哲学含义及其主语Ge-stell的提出}
		\paragraph{}
		stellen一词在《技术的追问》开篇并未被赋予哲学含义。海德格尔追问技术是从“本质”(Wesen)出发的。首先他对技术的本质和技术性的东西进行了区分,然后他首先直接否定了一种把技术视为中性物(etwas Neutrales)的见解。之后他给出了人们对于技术的两种通常见解:技术是合目的的工具(ein Mittel für Zwecke),或技术是人的行为(Tun des Menschen)。他将这些见解归结为“工具”,人借工具对物发生作用(Wirkung),由此引出“原因”(Ursache),并追溯到亚里士多德的四因说。海德格尔从希腊语的“原因”中辨读出了“招致”(Verschulden),之后以银盘为例解读“原因”与“招致”。在以银盘为例的论述中,海德格尔从“招致”推及“引发”(Ver-an-lassen)和“产出,生产,创造,带来”(Her-vor-bringen),指出“产出”从遮蔽状态而来,进入无蔽状态中。于是最终海德格尔将技术与“解蔽”紧密联系起来。达到“解蔽”概念后,海德格尔又从Technik的希腊语词源上进行分析。由此发现“技术”一词原初带有“艺术”、“创作”的含义,并与“认识”有很大关联。这些分析进一步证明了技术与解蔽的关系。
		\paragraph{}
		在中篇,海德格尔以大量举例的方式,以及在文中大量使用stellen词族的方式,将负面含义灌注于stellen这个词语之中。从34段到41段,海德格尔提出,现代技术中起支配作用的解蔽是“促逼”(Herausfordern),它向自然提出(stellen)蛮横要求。此后,海德格尔举例说明,现代的采矿业、食品工业、工厂、发电厂、旅游工业都处处体现着促逼意义上的摆置、订造的特征。海德格尔呈现的是一种连锁式的摆置、预订关系。在采矿业中,土地为矿石而被摆置,矿石举例来说为铀而被摆置,铀为毁灭或和平的目的而被释放。煤因为其中蕴藏的太阳的热量而被预订,太阳的热量为热能而被预订,热能提供蒸汽,蒸汽为驱动装置提供压力,驱动装置维持工厂运转。这种连锁关系中的“stellen”就是被提供,被放置并准备好,和进一步提供进而提出。被订造的东西被海德格尔命名为“持存物”(Bestand),海德格尔举了客机一例来说明持存物是如何处处被订造,又为下一级的订造随时做好准备的。
		\paragraph{}
		在这几段文字中,与stellen一词有着紧密关联的词语主要有Herausfordern和Entbergen,以及后来海德格尔提出的Bestand,从这些词语与stellen的关系中,可以从不同侧面认识stellen被赋予的哲学含义。Entbergen是技术的实质,但这不算其绝无仅有的特征,而事实上描述stellen的词语是Herausfordern(促逼)。促逼是先于stellen提出的,而且Herausfordern又是海德格尔用来直接形容stellen的,这种促逼向自然提出要求,摆置自然。对自然的摆置是促逼意义上的摆置。体现它们之间关系的句子有34段的一句。Das in der modernen Technik waltende Entbergen ist ein Herausfordern, das an die Natur das Ansinnen stellt… 第36段Das Stellen, das die Naturenergien herausfordert…以及第38段Das Entbergen, das die moderne Technik durchherrscht, hat den Chatakter des Stellens im Sinne der Herausforderung.现代技术的解蔽是一种促逼,摆置是向自然能源的促逼,解蔽拥有促逼意义上的摆置之特征。和stellen有着二级关联的词语包括Steuerung和Sicherung,控制和保障是技术的“stellen”这种解蔽方式(Entbergen)的主要特征。
		\paragraph{}
		从42段开始,海德格尔探讨了stellen的主语,即谁实行这种摆置的问题。直到45段得出初步的否定式结论,这不是纯粹的人的行为。在这过程中,海德格尔又举了几个例子,又使用了多次stellen的动词形式。但主要体现Stellen含义的是几次对同类句子的重复。这包括45段和48段的几句:das Wirkliche (Sichentbergende) als Bestand zu bestellen.海德格尔探究这些句子中动词bestellen的主语如果不是人,那么会是什么。
		\paragraph{}
		紧接着海德格尔由stellen词族在48段引出了Ge-stell(座架)这个著名的概念。在51段海德格尔终于在句子中明确用上了做出stellen动作的这个主语Ge-stell,对上文类似的话进行了重复。Ge-stell heißt das Versmmelnde jenes Stellens, das den Menschen stellt, d.h. herausfordert, das Wirkliche in der Weise des Bestellens als Bestand zu entbergen.后文中海德格尔也解释了Gestell词干中体现的stellen之含义。首先,前缀“Ge-”意味着它是stellen的集合(51段),是stellen词族中各个词语含义的集合,其次这里的词干保持着与它由之而来的另一种Stellen的相似,即Her- und Dar-stellen(产出和呈现)(52段)。它们让海德格尔联系到了“创作”这个希腊词语,进而联系到了无蔽,在Ge-stell中发生(sich ereignen)着无蔽状态。	
		
	\subsection{中、英译文解读}
		\paragraph{}
		正如前文所述,在《技术的追问》中,stellen的含义非常丰富,词族庞大,有动作性,也能表现位置性,与stehen Bestand 有一定关联。在其他的语言中是不太可能找到能够代替的一个词语的,用其他语言中多个相互关系不大的词语翻译stellen词族中词义不同却关联紧密的各个词语,必然使得表意大打折扣。但即便如此,借鉴中、英文翻译者堆某些词语的翻译方式依然能为加深理解提供帮助。
		\paragraph{}
		William Lovitt的英译本《The Question Concerning Technology and Other Essays》\cite{Quest_Concern_Tech}是《技术的追问》较为权威的英译版本。英译文将stellen翻译为to set或to set upon,有时被译为to supply,将bestellen译为order或on call。但这些词语远不能表达stellen的完整含义,因此译者写了详细的译注:
		\paragraph{}
		The verb stellen (to place or set) has a wide variety of uses. It can mean to put in place, to order, to arrange, to furnish or supply, and, in a
military context, to challenge or engage.\cite[pg. 15]{Quest_Concern_Tech}
		\paragraph{}
		对于Ge-stell,Lovitt译为Enframing,突出了Gestell的支架原义,而不刻意与stellen(set on, place)相呼应。
		\paragraph{}
		国内对《技术的追问》的翻译主要有孙周兴教授,事实上他的译本除了依据海德格尔原文外,还重点参考了Lovitt的英译本。
		\paragraph{}
		在孙周兴的文章《学术翻译的几个原则》\cite{Aca_Transl_SZx}中,孙周兴提出了语境原则、硬译原则、统一原则、可读原则这四项翻译方针。这其中,重点是硬译,即“宁取字面义,勿取解释义”,他还说道:哲学-思想类的译文就要蓄意地做得硬梆梆的,让一般的人们看不懂——因为原著本身就不是一般人所能接近的。若是把学术品也译得喜闻乐见……则学术翻译的意义已经丧失了大半。
		\paragraph{}
		以硬译为原则,孙周兴将stellen译为“摆置”,将bestellen译为“订造”,将Ge-stell译为“座架”。有些学者将stellen译为“拷打”,将Ge-stell译为“集置”,孙周兴与之形成鲜明对比。孙周兴的译法不直接揭示词语含义,是考虑到几乎无法找到能表达stellen或Ge-stell多方面含义的中文词语,与其用片面、夸张、生造的词语,不如硬译,让其哲学含义在行文中被揭示。
		
\section{Wesen(本质)及相关词语分析}
	\subsection{Wesen相关词语的词源}
		\paragraph{}
		《技术的追问》文中对本质一词已经进行了词源的探究,但是他的某些步骤将一词与另一词联系活不够严谨,或只是一笔带过。因此这里研究Wesen相关词语的词源,将囊括wesen、essensia、währen、Wahrheit、gewähren等词,将起到补充、印证或质疑海德格尔原文的作用。
		\paragraph{}
			\subparagraph{wesen}
			该词可追溯到8世纪的1古高地德语的wesan,在那时表示“在某地的停留(Verweilen),留存、存在、生存(Dasein)”。该词名词形式来自不定式动词,它属于sein范畴下的强变化动词。后来还发展出了居所、物、状态等含义。
			\subparagraph{währen}意为dauern、Bestand haben、 verweilen即持续,同时有存在之义。während作为弱变化持续体动词,是由名词Wesen对应的强变化动词wesen形成的。而且从历史词义上看,它们也有共同之处。
			\subparagraph{wahr, Wahrheit, wahren}
			währen的词形很容易让人联想到wahr(真的)、Wahrheit(真理、真相)和wahren(保护维护、维持)。在词源词典中,wahr和wahren是分开的,而Wahrheit归属于wahr的子条目。
wahr在古高地德语中就有“真(wirklich)”“确凿(gewiß)”的意思。Wahrheit属于wahr的派生词,在8或9世纪就有warheit。在wahr词条中,还提到了gewähren。wahren在古高地德语中表示结盟的忠实、保护、契约、保障等含义。词源词典中把wahren与wahr分开为两个词条进行解释,但这两个词的关系较为暧昧,在古高地德语、中古高地德语中古低地德语等中有大量词形非常相似的拼写方式。wahren与古高地德语的wahr有亲缘关系,不过在任何时期内wahr与wahren都没有等同过。但不论如何,wahr与wahren与währen从词源上没有直接联系
			\subparagraph{gewähren}
			该词在9世纪的古高地德语中就已经独立出现,意为Gewähr leisten, zugestehen,而Gewähr就是保障、担保之意,与wahr在古高地德语中的意思相近。gewähren与währen反倒没有什么关系。
			\subparagraph{essentia}
			德文作die Essenz,直译为Wesen,Wesenheit,词义为“精华、核心”,也有化学的浓缩物之义。essentia是拉丁文,意为“本质”,它的来源是esse,即是,存在,译为sein。可以看到,拉丁文和德文在“存在”和“本质”这里有巨大的相似性,都是从最基本的“是动词”产生出本质,即海德格尔所引用的“所是的那个什么”。
	\subsection{海德格尔对Wesen的用法}
		\paragraph{}
		海德格尔对Wesen的使用在全文中可谓从始至终。主要可以分为两大部分,以83段为分界。之前一直以通俗(geläufig)的含义上使用本质一词,而后对本质一词进行真正的探索。
		\paragraph{}
		海德格尔对本质一词的初步解释在开篇就给出了。一方面,海德格尔通过树的例子解释本质,他用了两个词,贯穿支配(durch walten)。贯穿描述了一种共性,即每棵树都被本质所支配,而支配指出,本质不是一颗树,也不是树的某些共性特征,而是高于树的一个抽象概念,它使得这一棵东西成为树(这种抽象对后面的“座架也同样适用”)。另一方面依据一种古老的学说(nach alter Lehre),本质被看作某物所是的那个“什么”(was etwas ist)。这些便是本质的“通俗”含义。
		\paragraph{}
		即使是在通俗的含义上使用本质一词,海德格尔在前文中的这个词也有多种不同用法。从本质的主语进行分析,最主要的当然就是技术(Wesen der Technik),第59、68、70段明显地重复着一句话:技术的本质居于(beruhen)座架之中。凡是说到技术的本质,海德格尔讲的都是座架,而暂不涉及本质这个词本身,比如78段,技术之本质作为解蔽之命运乃是危险。这里说“本质乃是危险”实际是说座架的危险性质。
		\paragraph{}
		除了以技术作为本质的主语,海德格尔还在文中以历史、自由、人做为本质的主语。而这三个主语直接的关系体现在海德格尔的推演过程中。技术的本质将人带上一条路,它将现实(Wirkliche)转变为Bestand。带上路这种遣送(Schicken)是命运(Geschick),而命运规定着的就是历史(Geschichte)的本质。(63段)。紧接着海德格尔谈及自由,人归属于命运领域,成为倾听者,才是自由的,自由的本质不是意志,自由掌管着开放领域,这就联系到了解蔽概念。通过这一系列的推演,海德格尔得以拓展技术的本质中的“解蔽”内涵,探究这是如何的一种解蔽。之后海德格尔用到了“人”作为本质的主语,因为技术的本质指点的解蔽是一种stellen的单一形式的解蔽,一种贬义的解蔽,解蔽技术性的同时,人的本质却被遮蔽了,人不再碰到自身。这便是危险,座架的危险在人的本质处触动了人类。
		\paragraph{}
		海德格尔还用到了3次本质的动词形式wesen。其中两次与技术或座架有关,一次是在介绍柏拉图中使用。31段提到:技术在真理发生的领域现身(west)。62段提到,我们是否和如何真正投入到座架本身现身(west)于其中的那个东西中。这两个地方,包括介绍柏拉图的一个地方,wesen都与一个介词in共同出现,wesen都发生在一个领域之中,某物在一个领域中成其本质。在这些上下文中,如果把wesen改成sein,句子也可以说得通,从词源上我们也有依据,允许我们如此理解。
		\paragraph{}
		从对本质一词的描述来看,海德格尔始终说要揭开(enthüllen, ans Licht zu heben)本质,但从历史上看,本质却如纱巾(Schleier)遮蔽(verhüllen, verbergen)着自身。因此57段说,现代技术虽然晚于自然科学的发展,但技术的本质则是历史上(geschichtlich)早先的东西。
		\paragraph{}
		
	\subsection{Wesen相关词语的解读与关联性分析}
		\paragraph{}
		到了83段以后,Wesen一词被海德格尔重点探究。首先海德格尔否定了对Wesen的一些认识。84段重复了开头第4段的定义,从哲学的学院语言中,本质的意思是was etwas ist,某物所是的那个“什么”,即Washeit。接下来海德格尔由之前的通俗或学院语言中的本质而上升了一个高度,并否定了本质的种类之说。海德格尔使用Weise一词解释技术的座架本质Das Ge-stell ist eine geschickhafte Weise des Entbergens, nämlich das herausfordernde… Aber diese Weisen sind nicht Arten… (84段)。85段得出否定性结论,座架不是种类(Gattung)或essentia意义上的技术本质,这里的essentia的历史由来在本论文前面的词源研究中提及,essentia意义上的本质侧重于拉丁文词干esse,即“是,存在”,与上述“学院语言”的本质含义相一致。接下来海德格尔开始从Weise出发,从另一种意义上去解释“本质”。
		\paragraph{}
		接下来海德格尔从语用和词源角度分析了Wesen一词。延续上文提出的Weise,举例“家政”“国体”来说明,用walten, sich verwalten, entfalten, verfallen几个词进一步解释了Weise。这里的这些动词在句中位置如下:die Weise, wie Haus und Staat walten… 而后面一句用了完全相同的句式:Es ist die Weise, wie sie wesen.海德格尔用上述的四个词来解释了动词wesen。由动词发展出名词Wesen,而后海德格尔从词源的角度将Wesen与währen相联系。又由苏格拉底和柏拉图的思考,印证了Wesende带有的Währende的意义,并且他们的Währende是作为Fortwährende来进行思考的,而对Fortwährende的寻找是在那些作为Bleibende坚持于一切出现之物的东西中进行的,对于他们来说这就是观念(Idee)。
	\subsection{座架(Ge-stell)作为技术本质}
	\subsection{中、英译文解读}
\section{结语}
	\subsection{总结}
	\subsection{展望}
	\subsection{技术的座架本质对技术时代的启示}
\renewcommand\refname{参考文献}
\bibliographystyle{plain}
\bibliography{citations.bib}
\end{document} 