\documentclass{article} 
\usepackage{xeCJK}
\usepackage{cite}
\setCJKmainfont{Droid Sans Fallback}
\author{叶欣} 

\title{《技术的追问》词语分析} 
\begin{document}   
\maketitle   



\begin{abstract}

\end{abstract}



\tableofcontents

\section{引言} 

	\subsection{选题的意义} 
		\paragraph{}
海德格尔是20世纪最伟大的思想家之一,他对技术问题的探究是他后期思想的重要组成部分。针对技术本质的问题,海德格尔提出了座架(Gestell)这一概念,并在海德格尔哲学的语境下进行了深入探索。由于海德格尔语言特点略显晦涩,其写作中“座架”“本质”等概念较为抽象,有时词义被海德格尔拓展,有时词语被海德格尔溯源,并与其他词语相关联,往往令人难以理解。因此本研究认为,从分析德文词语词源、基本词义、词族关联等方面入手,是理解海德格尔技术之思的重要途径。
		\paragraph{}
海德格尔哲学,尤其是他关于技术的论述引发了中国学界的重视。在译介方面,以孙周兴为代表的学者翻译了海德格尔大量演讲与论文,对海德格尔思想表述中Ereignis、Dasein、Gestell等重点概念的含义和译法进行了深入探讨。在对海德格尔技术思想的解读方面,国内已有大量论著,如范玉刚的《睿思与歧误:一种对海德格尔技术之思的审美解读》,包国光的《海德格尔生存论视域下的技术》,研究论文更是多如牛毛。但是,从语言、词语的角度出发,对海德格尔技术思想重点概念所进行的研究却尚未看到,这不得不说是一个遗憾。
		\paragraph{}
此外,本研究尝试结合中、英文翻译用词进行研究,从词语角度辨析多种语言对海德格尔哲学的各自解读,能极大帮助我们理解德文词语的完整含意,并认识不同文化对海德格尔哲学的引介情况。
	\subsection{问题的提出}
		\paragraph{}
     从词语角度出发,我们提出以下问题。
        
		\subparagraph{1} 
            在海德格尔思想中,为何用stellen(摆置)解读现代技术?
		\subparagraph{2} 
           “技术的本质”中“本质”(Wesen)的含义是什么?技术的本质如何与技术性的东西区分?
		\subparagraph{3}海德格尔为何提出Gestell(座架)一词?它在何种意义上是技术的本质?
		\subparagraph{4}中文、英文对德文词语的翻译能否为我们理解海德格尔提供线索?
		\subparagraph{5}技术的座架本质对身处技术时代的我们有何启示?
	\subsection{研究目的}
		\paragraph{}
        本课题拟通过对海德格尔《技术的追问》中核心词汇的深入解读,通过对Gestell(座架)和Wesen(本质)等词汇的词源、所属词族、中英文翻译的研究,结合海德格尔相关文章的用词背景、在本文中对词语关系的推演,以达到认识德语词语的完整含义、追踪海德格尔技术之思的路途、深入理解海德格尔所谓“技术本质”的目的,并为技术时代的生活提供借鉴。
	\subsection{论文结构}
		\paragraph{}
        论文的第一部分为引言,介绍本课题的研究意义、目的,提出待研究的问题。
		\paragraph{}
        论文的第二部分,简要介绍海德格尔哲学思想、语言特点,对本课题研究对象《技术的追问》进行总体介绍,并对《技术的追问》中的重点词语及所属词族进行统计和分类,便于后续深入研究的开展。
        \paragraph{}
        论文的第三部分,拟分析stellen(摆置)这一核心词族的词源、用法和含义。理解海德格尔对stellen的论述,认识Gestell(座架)概念如何被提出。结合中英译文的处理方式进行理解。
        \paragraph{}
        论文的第四部分,拟分析“座架作为技术之本质(Wesen)”中,本质一词的词源、用法和含义。区分技术本质与技术性的东西。探索währen、gewähren、Wahrheit等词语是如何从德语角度和海德格尔的角度与Wesen联系起来的。揭示座架何以是技术之本质。
        \paragraph{}
        论文的第五部分为结语,对本文进行简要总结。还会提及海德格尔对其他词语的运用,作为对未来研究的展望。最后,反思技术的座架本质对技术时代的启示。
\section{海德格尔与《技术的追问》} 
	\subsection{海德格尔的技术思想}
		\paragraph{}
	\subsection{海德格尔的语言特点}
		\paragraph{}
		从海德格尔的多篇著述、各国学者对海德格尔作品的翻译、他人对海德格尔的评价来看,可以发现海德格尔的语言有其鲜明特点,主要包括:方向性强、严整性弱、重视词语的使用等。
		\paragraph{}
		海德格尔作品的方向性、顺序性强。其行文过程如同一次探索的经历。正如海德格尔在《技术的追问》开篇所说,“追问构筑一条道路。因此之故,我们大有必要首先关注一下道路,而不要牵挂于个别的句子和明目。”
\cite{slct_Hei_Szx}
基于这样一个基调,在海德格尔行文过程中,一个词语可以在文中多次出现,但后来出现时往往比之前含义更加丰富。比如在《技术的追问》中,本质一词开头就多次出现,但这时“技术本质”的含义只是以“某物所是的那个什么”这个程度上理解的,直到文章中后部,出现一句“直到现在,我们还是在流俗的含义上来理解‘本质’一词的”
\cite{slct_Hei_Szx}
,从这开始,本质一词的含义才被继续开发,并联系到“持续”“允诺”等词语。
在海德格尔的全集前言草稿中,他写道:“Wege – nicht Werke”(道路——而非著作)。在对此的解释中,他说他的全集是“在对多义的存在问题所作的变动不居的追问道路之野上的一种行进(Unter-wegs,在途中)”。
\cite{sprach_sein_Szx}
还可以注意到,道路一词使用了复数形式,这正印证了他的追问的“变动不居”,即海德格尔不认为他的思想会像古典哲学那样搭建一座形而上学的完整建筑,而是从多个方向探索,行进在思想道路上。这与海德格尔的教学习惯有密切联系,他不希望学生们只是听从他的观点而已。他希望学生或读者在思之旅途上伴随他,引领学生修筑自己的思想道路。
		\paragraph{}
		正是由于海德格尔语言如同道路一般的特点,他的文章有时令人觉得不够严谨。伽达默尔说,“后期海德格尔自己为了逃避形而上学的语言而发展出他半诗性的特殊语言”
\cite{Text_Explain_DeFr}
,他甚至用“女巫式风格”来形容其后期著作
\cite[pg. 113]{Deonstr_DeFr}
。这一点在《技术的追问》中也有所体现。比如从währen(持续)到gewähren(允诺),海德格尔仅举歌德作品中一例用词,说歌德的“耳朵在此听出了两词之间的未曾道出的一致”,便下了结论“只有允诺者才持续”
。又比如他所说危险中的解救,是以“荷尔德林的诗句道出了真理”为前提的\cite{slct_Hei_Szx}。
海德格尔有时也以比喻代替论证,比如在《技术的追问》结尾使用的Konstellation(星座)一词。
		\paragraph{}
		对词语的重视是海德格尔思想的重要特点。
\renewcommand\refname{参考文献}
\bibliographystyle{plain}
\bibliography{citations.bib}
\end{document} 